% --- Halaman Judul ---
\begin{titlepage}
    % === KODE UNTUK FULL PAGE IMAGE COVER ===
    \AddToShipoutPictureBG*{%
        \put(0,0){%
            \parbox[b][\paperheight]{\paperwidth}{%
                \vfill
                \includegraphics[width=\paperwidth,height=\paperheight]{cover.png}
            }%
        }%
    }
    % ========================================

    % Pengaturan Rata Kiri & Font
    \raggedright
    \sffamily 
    
    % 1. Tulisan "Innovation Paper"
    {\fontsize{12}{13}\selectfont \textbf{INNOVATION PAPER}} \\[0.5cm]
    
    % 2. Tulisan "Judul" 
    % PERBAIKAN: Mengubah {42} menjadi {38} agar jarak baris judul lebih rapat.
    {\fontsize{36}{25}\selectfont \textbf{JUDUL PENELITIAN ATAU INOVASI DITULIS DI SINI DENGAN HURUF KAPITAL}\par} 
    \vspace{1.5cm}

    % 3, 4, 5, 6 & Konfigurasi Baru (DE Brief BSN)
    % PERBAIKAN: Menggunakan {12}{13.5} agar teks lebih padat (single spacing ketat).
    \noindent
    \begin{minipage}[t]{0.55\textwidth} % Kolom Kiri
        {\fontsize{12}{13.5}\selectfont 
        Dipersembahkan oleh:\\[0.3cm] % Jarak diperkecil (0.5cm -> 0.3cm) agar pas "1x enter"
        \textbf{\underline{NAMA PENULIS / TIM}} \\ 
        DIVISI / UNIT KERJA 
        }
    \end{minipage}%
    \hspace{0.5cm} 
    \begin{minipage}[t]{0.4\textwidth} % Kolom Kanan
        {\fontsize{12}{13.5}\selectfont 
        Untuk\\[0.3cm] % Jarak disamakan
        \textbf{NAMA STAKEHOLDER / TUJUAN}
        }
    \end{minipage}
    \vfill
\end{titlepage}