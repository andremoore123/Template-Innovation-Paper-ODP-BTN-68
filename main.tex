% ================================================================
% 1. SETUP & PREAMBLE (SESUAI TEMPLATE ANDA)
% ================================================================
% Menggunakan class 'book' agar support \chapter
\documentclass[12pt, a4paper, oneside]{book} 

% --- MARGIN ---
% Margin Bawah (bottom) diubah menjadi 5.5cm sesuai kode Anda
\usepackage[left=3cm, right=3cm, top=4cm, bottom=5.5cm]{geometry}

\usepackage[utf8]{inputenc}
\usepackage[bahasa]{babel}
\usepackage{indentfirst} % Spasi untuk paragraf pertama
\usepackage{sectsty}     % Untuk styling judul section
\usepackage{graphicx}
\usepackage{xcolor}
\usepackage{tikz}
\usepackage{eso-pic}
\usepackage{fancyhdr}
\usepackage{enumitem}
\usepackage{natbib}      % Untuk Daftar Pustaka
\usepackage[bookmarks,hypertexnames=false,debug]{hyperref}
\usepackage{titletoc}
\usepackage{tabularx}
\usepackage{etoolbox}
\usepackage{colortbl}    % Untuk warna tabel gantt chart

% --- PENGATURAN JUDUL BAB & SUB-BAB (Sesuai Template) ---
\usepackage[compact]{titlesec} 

% --- KONFIGURASI DAFTAR GAMBAR & TABEL (TITLETOC) ---
% 1. Menghilangkan jarak antar bab (vertical space)
\makeatletter
\patchcmd{\@chapter}{\addtocontents{lof}{\protect\addvspace{10\p@}}}{}{}{}
\patchcmd{\@chapter}{\addtocontents{lot}{\protect\addvspace{10\p@}}}{}{}{}
\makeatother

% 2. Mengatur format entry agar tidak ada indentasi (Rata Kiri)
% [0pt] : Mengatur indentasi kiri menjadi 0 (menempel margin kiri)
\titlecontents{figure}[0pt]
  {} 
  {\thecontentslabel\hspace{1em}} % Format label nomor (misal: 1.1) + spasi
  {}
  {\titlerule*[0.5pc]{.}\contentspage} % Garis titik-titik + nomor halaman

\titlecontents{table}[0pt]
  {}
  {\thecontentslabel\hspace{1em}}
  {}
  {\titlerule*[0.5pc]{.}\contentspage}
  
% Font khusus untuk Chapter (Tengah, Besar, Bold)
\newcommand{\bigsize}{\fontsize{16pt}{14pt}\selectfont}
\titleformat{\chapter}[display]
  {\centering\bigsize\bfseries} % Format teks
  {\chaptertitlename\ \thechapter} % Label "BAB I"
  {10pt} % Jarak antara "BAB I" dan "JUDUL"
  {\bigsize} % Format judul

% Mengatur Spasi/Padding Chapter (Atas dan Bawah)
% \titlespacing*{command}{left}{before-sep}{after-sep}
\titlespacing*{\chapter}{0pt}{-20pt}{20pt} 

% Styling Section (Mirip template: Large, Bold)
\sectionfont{\large\bfseries}

% Penomoran Romawi untuk Chapter (I, II, III)
\renewcommand{\thechapter}{\Roman{chapter}}
\renewcommand\thesection{\arabic{chapter}.\arabic{section}}
\renewcommand\thesubsection{\thesection.\arabic{subsection}}
\renewcommand{\theequation}{\arabic{chapter}.\arabic{equation}}
\renewcommand{\thefigure}{\arabic{chapter}.\arabic{figure}}
\renewcommand{\thetable}{\arabic{chapter}.\arabic{table}}

% Nama Daftar Pustaka
\renewcommand\bibname{Daftar Pustaka}
\addto{\captionsbahasa}{\renewcommand{\bibname}{Daftar Pustaka}}

% --- SPACING KONTEN ---
\linespread{1} 

% ================================================================
% 2. KONFIGURASI HEADER/FOOTER (LAYOUT BTN)
% ================================================================

% --- Background ---
\newcommand\BackgroundPic{
    \put(0,0){
        \parbox[b][\paperheight]{\paperwidth}{%
            \vfill
            \centering
            \IfFileExists{background.png}{
                % MODIFIKASI: Width 80% dari kertas, Height tetap full kertas
                \includegraphics[width=0.2\paperwidth,height=0.4\paperheight]{background.png}
            }{
                \begin{tikzpicture}
                    \fill[gray!5] (0,0) rectangle (\paperwidth,\paperheight);
                \end{tikzpicture}
            }
            \vfill
        }
    }
}
% CATATAN: AddToShipoutPicture dihapus dari sini agar tidak muncul di Cover.
% Perintah ini akan dipanggil setelah Cover selesai.

% --- Header & Footer ---
% Kita gunakan 'fancy' untuk semua halaman (termasuk halaman awal Bab)
\pagestyle{fancy}
\fancyhf{} 
\renewcommand{\headrulewidth}{0pt}
\setlength{\headheight}{50pt}

% Paksa halaman awal Bab (yang biasanya 'plain') menjadi 'fancy' agar header tetap muncul
\fancypagestyle{plain}{
  \fancyhf{}
  \renewcommand{\headrulewidth}{0pt}
  \fancyhead[C]{%
    \begin{tikzpicture}[remember picture, overlay]
        \node[anchor=north, inner sep=0pt] at (current page.north) {
            \IfFileExists{header.png}{
                \includegraphics[width=\paperwidth]{header.png} 
            }{
                \fbox{\parbox{\textwidth}{\centering \textbf{[HEADER GAMBAR]} \\ \small Upload \texttt{header.png}}}
            }
        };
    \end{tikzpicture}
  }
  \fancyfoot[C]{%
    \begin{tikzpicture}[remember picture, overlay]
         \node[anchor=south, inner sep=0pt] at (current page.south) {
             \IfFileExists{footer.png}{
                 \includegraphics[width=\paperwidth]{footer.png}
             }{
                 \fbox{\parbox{\textwidth}{\centering \textbf{[FOOTER GAMBAR]} \\ \small Halaman \thepage}}
             }
         };
    \end{tikzpicture}
  }
}

% Setting default fancy (untuk halaman isi lainnya)
\fancyhead[C]{%
    \begin{tikzpicture}[remember picture, overlay]
        \node[anchor=north, inner sep=0pt] at (current page.north) {
            \IfFileExists{header.png}{
                \includegraphics[width=\paperwidth]{header.png} 
            }{
                \fbox{\parbox{\textwidth}{\centering \textbf{[HEADER GAMBAR]} \\ \small Upload \texttt{header.png}}}
            }
        };
    \end{tikzpicture}
}

\fancyfoot[C]{%
    \begin{tikzpicture}[remember picture, overlay]
         \node[anchor=south, inner sep=0pt] at (current page.south) {
             \IfFileExists{footer.png}{
                 \includegraphics[width=\paperwidth]{footer.png}
             }{
                 \fbox{\parbox{\textwidth}{\centering \textbf{[FOOTER GAMBAR]} \\ \small Halaman \thepage}}
             }
         };
    \end{tikzpicture}
}

% List spacing adjustments
\setlist[itemize]{label=-}

% ================================================================
% 3. DOKUMEN UTAMA
% ================================================================
\begin{document}

% --- Halaman Judul ---
\AddToShipoutPicture{\BackgroundPic}
% --- Halaman Judul ---
\begin{titlepage}
    % === KODE UNTUK FULL PAGE IMAGE COVER ===
    \AddToShipoutPictureBG*{%
        \put(0,0){%
            \parbox[b][\paperheight]{\paperwidth}{%
                \vfill
                \includegraphics[width=\paperwidth,height=\paperheight]{cover.png}
            }%
        }%
    }
    % ========================================

    % Pengaturan Rata Kiri & Font
    \raggedright
    \sffamily 
    
    % 1. Tulisan "Innovation Paper"
    {\fontsize{12}{13}\selectfont \textbf{INNOVATION PAPER}} \\[0.5cm]
    
    % 2. Tulisan "Judul" 
    % PERBAIKAN: Mengubah {42} menjadi {38} agar jarak baris judul lebih rapat.
    {\fontsize{36}{25}\selectfont \textbf{JUDUL PENELITIAN ATAU INOVASI DITULIS DI SINI DENGAN HURUF KAPITAL}\par} 
    \vspace{1.5cm}

    % 3, 4, 5, 6 & Konfigurasi Baru (DE Brief BSN)
    % PERBAIKAN: Menggunakan {12}{13.5} agar teks lebih padat (single spacing ketat).
    \noindent
    \begin{minipage}[t]{0.55\textwidth} % Kolom Kiri
        {\fontsize{12}{13.5}\selectfont 
        Dipersembahkan oleh:\\[0.3cm] % Jarak diperkecil (0.5cm -> 0.3cm) agar pas "1x enter"
        \textbf{\underline{NAMA PENULIS / TIM}} \\ 
        DIVISI / UNIT KERJA 
        }
    \end{minipage}%
    \hspace{0.5cm} 
    \begin{minipage}[t]{0.4\textwidth} % Kolom Kanan
        {\fontsize{12}{13.5}\selectfont 
        Untuk\\[0.3cm] % Jarak disamakan
        \textbf{NAMA STAKEHOLDER / TUJUAN}
        }
    \end{minipage}
    \vfill
\end{titlepage}

% --- Front Matter (Dihilangkan sesuai request) ---
\frontmatter 
\pagenumbering{roman}
\phantomsection
\input{Abstrak}
\phantomsection
\addcontentsline{toc}{chapter}{Daftar Isi}
\tableofcontents

\phantomsection
\addcontentsline{toc}{chapter}{Daftar Gambar}
\listoffigures

\phantomsection
\addcontentsline{toc}{chapter}{Daftar Tabel}
\listoftables

% --- Main Content ---
\mainmatter
\pagenumbering{arabic}
\chapter{Pendahuluan}
\section{Latar Belakang}
Lorem ipsum dolor sit amet, consectetur adipiscing elit. Nullam in dui mauris. Vivamus hendrerit arcu sed erat molestie vehicula. Sed auctor neque eu tellus rhoncus ut eleifend nibh porttitor. Ut in nulla enim. Phasellus molestie magna non est bibendum non venenatis nisl tempor. Suspendisse dictum feugiat nisl ut dapibus \cite{dummy1}.

Vivamus hendrerit arcu sed erat molestie vehicula. Sed auctor neque eu tellus rhoncus ut eleifend nibh porttitor. Ut in nulla enim. Phasellus molestie magna non est bibendum non venenatis nisl tempor. Suspendisse dictum feugiat nisl ut dapibus. Mauris iaculis porttitor posuere. Praesent id metus massa, ut blandit odio. Proin quis tortor orci. Etiam at risus et justo dignissim congue.

\section{Rumusan Masalah}
Berdasarkan latar belakang di atas, rumusan masalah yang diajukan adalah:
\begin{enumerate}
    \item Lorem ipsum dolor sit amet, consectetur adipiscing elit?
    \item Sed do eiusmod tempor incididunt ut labore et dolore magna aliqua?
    \item Ut enim ad minim veniam, quis nostrud exercitation ullamco laboris nisi ut aliquip ex ea commodo consequat?
\end{enumerate}

\section{Tujuan}
Tujuan dari penulisan dokumen ini adalah:
\begin{enumerate}
    \item \textbf{Tujuan Pertama:} Lorem ipsum dolor sit amet, consectetur adipiscing elit.
    \item \textbf{Tujuan Kedua:} Sed do eiusmod tempor incididunt ut labore et dolore magna aliqua.
    \item \textbf{Tujuan Ketiga:} Ut enim ad minim veniam, quis nostrud exercitation ullamco laboris.
\end{enumerate}
\chapter{Analisa Inovasi}
\section{Landasan Teori}
\subsection{Teori A}
Lorem ipsum dolor sit amet, consectetur adipiscing elit. Integer nec odio. Praesent libero. Sed cursus ante dapibus diam. Sed nisi. Nulla quis sem at nibh elementum imperdiet. Duis sagittis ipsum. Praesent mauris. Fusce nec tellus sed augue semper porta. Mauris massa \cite{dummy2}.

\subsection{Teori B}
Class aptent taciti sociosqu ad litora torquent per conubia nostra, per inceptos himenaeos. Curabitur sodales ligula in libero. Sed dignissim lacinia nunc. Curabitur tortor. Pellentesque nibh. Aenean quam. In scelerisque sem at dolor. Maecenas mattis. Sed convallis tristique sem.

\section{Deskripsi Sistem}
Lorem ipsum dolor sit amet, consectetur adipiscing elit. Vivamus lacinia odio vitae vestibulum vestibulum. Cras venenatis euismod malesuada. Nulla facilisi. Donec sed odio dui. Nulla vitae elit libero, a pharetra augue.

\section{Perencanaan (Timeline)}

\subsection{Gantt Chart}
Berikut adalah rencana kerja yang diusulkan:

\begin{table}[h]
\centering
\caption{Contoh Gantt Chart}
\label{tab:gantt_dummy}
\begin{tabular}{|l|c|c|c|c|}
\hline
\textbf{Kegiatan} & \textbf{Mg 1} & \textbf{Mg 2} & \textbf{Mg 3} & \textbf{Mg 4} \\ \hline
1. Perencanaan & \cellcolor{gray!50} & & & \\ \hline
2. Analisis & & \cellcolor{gray!50} & & \\ \hline
3. Implementasi & & & \cellcolor{gray!50} & \\ \hline
4. Pengujian & & & & \cellcolor{gray!50} \\ \hline
\end{tabular}
\end{table}

\subsection{Matriks Tanggung Jawab (RACI)}
Distribusi tanggung jawab antar tim adalah sebagai berikut:

\begin{table}[h]
\centering
\caption{Contoh Matriks RACI}
\label{tab:raci_dummy}
\begin{tabularx}{\textwidth}{|X|c|c|c|}
\hline
\textbf{Aktivitas} & \textbf{Role A} & \textbf{Role B} & \textbf{Role C} \\ \hline
Task 1 & R & A & I \\ \hline
Task 2 & C & R & I \\ \hline
Task 3 & I & C & A/R \\ \hline
\end{tabularx}
\end{table}

\section{Analisa Biaya}
Lorem ipsum dolor sit amet, consectetur adipiscing elit. Sed do eiusmod tempor incididunt ut labore et dolore magna aliqua. Ut enim ad minim veniam, quis nostrud exercitation ullamco laboris nisi ut aliquip ex ea commodo consequat.

\chapter{Implementasi dan Kesimpulan}
Lorem ipsum dolor sit amet, consectetur adipiscing elit. Vivamus lacinia odio vitae vestibulum vestibulum. Cras venenatis euismod malesuada.

\begin{figure}[h]
    \centering
    \framebox{\parbox{0.9\textwidth}{\centering \vspace{3cm} \textbf{[PLACEHOLDER GAMBAR HASIL]} \\ \small (Contoh: Grafik, Diagram, Screenshot Sistem) \vspace{3cm}}}
    \caption{Contoh Gambar Hasil Implementasi}
    \label{fig:dummy_result}
\end{figure}

Duis aute irure dolor in reprehenderit in voluptate velit esse cillum dolore eu fugiat nulla pariatur. Excepteur sint occaecat cupidatat non proident, sunt in culpa qui officia deserunt mollit anim id est laborum.

\section{Kesimpulan}
Berdasarkan pembahasan di bab-bab sebelumnya, dapat disimpulkan:
\begin{enumerate}
    \item Lorem ipsum dolor sit amet, consectetur adipiscing elit.
    \item Sed do eiusmod tempor incididunt ut labore et dolore magna aliqua.
    \item Ut enim ad minim veniam, quis nostrud exercitation ullamco laboris nisi ut aliquip ex ea commodo consequat.
\end{enumerate}

\section{Saran}
Saran untuk pengembangan selanjutnya adalah:
\begin{itemize}
    \item Duis aute irure dolor in reprehenderit in voluptate velit esse cillum dolore.
    \item Excepteur sint occaecat cupidatat non proident, sunt in culpa qui officia deserunt.
\end{itemize}

% --- Back Matter ---
\backmatter
\addcontentsline{toc}{chapter}{Daftar Pustaka}
\bibliographystyle{plain} 
\bibliography{References}

\phantomsection
\chapter*{Lampiran}

\section*{Lampiran A: Data Tambahan}
% Jarak vertikal dikurangi sesuai request
\vspace{-20pt}

\begin{table}[h]
\centering
\caption{Contoh Tabel Data Lampiran} \label{tab:appendix_dummy}
\begin{tabular}{|c|l|l|}
\hline
\textbf{No} & \textbf{Parameter} & \textbf{Nilai} \\ \hline
1 & Parameter A & 100 \\ \hline
2 & Parameter B & 200 \\ \hline
3 & Parameter C & 300 \\ \hline
\end{tabular}
\end{table}

\section*{Lampiran B: Data Tambahan}
% Jarak vertikal dikurangi sesuai request
\vspace{-20pt}

\begin{table}[h]
\centering
\caption{Contoh Tabel Data Lampiran} \label{tab:appendix_dummy}
\begin{tabular}{|c|l|l|}
\hline
\textbf{No} & \textbf{Parameter} & \textbf{Nilai} \\ \hline
1 & Parameter A & 100 \\ \hline
2 & Parameter B & 200 \\ \hline
3 & Parameter C & 300 \\ \hline
\end{tabular}
\end{table}

\end{document}